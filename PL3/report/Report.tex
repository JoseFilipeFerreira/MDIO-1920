\documentclass[a4paper]{report}
\usepackage[utf8]{inputenc}
\usepackage[portuguese]{babel}
\usepackage{hyperref}
\usepackage{a4wide}
\hypersetup{pdftitle={MDIO - Trabalho 2},
pdfauthor={João Teixeira, José Ferreira, Miguel Solino},
colorlinks=true,
urlcolor=blue,
linkcolor=black}
\usepackage{subcaption}
\usepackage[cache=false]{minted}
\usepackage{listings}
\usepackage{booktabs}
\usepackage{multirow}
\usepackage{appendix}
\usepackage{tikz}
\usepackage{authblk}
\usepackage{bashful}
\usepackage{verbatim}
\usepackage{amsmath}
\usepackage{tikz}
\usepackage{tikz,fullpage}
\usetikzlibrary{arrows,%
                petri,%
                topaths}%
\usepackage{tkz-berge}
\usetikzlibrary{positioning,automata,decorations.markings}
\AfterEndEnvironment{figure}{\noindent\ignorespaces}
\AfterEndEnvironment{table}{\noindent\ignorespaces}

\begin{document}

\title{MDIO - Trabalho 2\\ 
\large Grupo Nº 3}
\author{João Teixeira (A85504) \and José Ferreira (A83683) \and Miguel Solino (A86435)}
\date{\today}

\begin{center}
    \begin{minipage}{0.75\linewidth}
        \centering
        \includegraphics[width=0.4\textwidth]{images/eng.jpeg}\par\vspace{1cm}
        \vspace{1.5cm}
        \href{https://www.uminho.pt/PT}
        {\color{black}{\scshape\LARGE Universidade do Minho}} \par
        \vspace{1cm}
        \href{https://www.di.uminho.pt/}
        {\color{black}{\scshape\Large Departamento de Informática}} \par
        \vspace{1.5cm}
        \maketitle
    \end{minipage}
\end{center}

\tableofcontents

\pagebreak

\chapter{Introdução}

\chapter{Parte 0}
A lista de atividade/duração/precedêndia apresentada inicialmente tem a seguinte
estrutura:
\begin{table}[H]
    \centering
    \begin{tabular}{|l|l|l|}
        \hline
        Atividade & Duração & Precedencia \\ \hline
        0         & 4       & -           \\ \hline
        1         & 6       & 0           \\ \hline
        2         & 7       & 1, 4        \\ \hline
        3         & 2       & 2, 5        \\ \hline
        4         & 9       & 0, 7        \\ \hline
        5         & 4       & 4, 8        \\ \hline
        6         & 5       & -           \\ \hline
        7         & 6       & 6           \\ \hline
        8         & 4       & 7, 10       \\ \hline
        9         & 2       & 8, 11       \\ \hline
        10        & 8       & 6           \\ \hline
        11        & 7       & 10          \\ \hline
    \end{tabular}
\end{table}
Que pode ser representada na forma de um grafo:
\begin{figure}[H]
\centering
\begin{tikzpicture}[scale=1,transform shape]
  \Vertex[x=0.5,y=6]{0}
  \Vertex[x=2,y=6]{1}
  \Vertex[x=4,y=6]{2}
  \Vertex[x=6,y=6]{3}

  \Vertex[x=-1,y=4]{ini}
  \Vertex[x=3,y=4]{4}
  \Vertex[x=5,y=4]{5}
  \Vertex[x=7,y=4]{fim}
  
  \Vertex[x=0.5,y=2]{6}
  \Vertex[x=2,y=2]{7}
  \Vertex[x=4,y=2]{8}
  \Vertex[x=6,y=2]{9}

  \Vertex[x=2,y=0]{10}
  \Vertex[x=4,y=0]{11}

  \tikzstyle{LabelStyle}=[fill=gray!30]
  \tikzstyle{EdgeStyle}=[post]
    \Edge(ini)(0)
    \Edge(0)(1)
    \Edge(1)(2)
    \Edge(2)(3)
    \Edge(3)(fim)
    \Edge(0)(4)
    \Edge(4)(2)
    \Edge(4)(5)
    \Edge(5)(3)
    \Edge(5)(fim)
    \Edge(ini)(6)
    \Edge(6)(7)
    \Edge(7)(4)
    \Edge(7)(8)
    \Edge(8)(5)
    \Edge(8)(9)
    \Edge(9)(fim)
    \Edge(6)(10)
    \Edge(10)(8)
    \Edge(10)(11)
    \Edge(11)(9)
\end{tikzpicture}
\end{figure}

Observando os números de inscrição dos membros do grupo, constatamos
que o maior pertencia ao aluno Miguel Solino (86435).
Fazendo \textit{pattern matching} para o padrão ABCDE e seguindo as
regras definidas no enunciado, concluímos que a orientação das ruas é:

\begin{enumerate}
    \item A é igual a 8;
    \item B é igual a 6;
    \item C é igual a 4;
    \item D é igual a 3;
    \item E é igual a 5;
\end{enumerate}
Assim, as atividades com o número 3 e 5 serão eliminadas 

\begin{table}[H]
    \centering
    \begin{tabular}{|l|l|l|}
        \hline
        Atividade & Duração & Precedência \\ \hline
        0         & 4       & -           \\ \hline
        1         & 6       & 0           \\ \hline
        2         & 7       & 1, 4        \\ \hline
        %3         & 2       & 2, 5        \\ \hline
        4         & 9       & 0, 7        \\ \hline
        %5         & 4       & 4, 8        \\ \hline
        6         & 5       & -           \\ \hline
        7         & 6       & 6           \\ \hline
        8         & 4       & 7, 10       \\ \hline
        9         & 2       & 8, 11       \\ \hline
        10        & 8       & 6           \\ \hline
        11        & 7       & 10          \\ \hline
    \end{tabular}
\end{table}
Que pode ser representada na forma de um grafo:
\begin{figure}[H]
\centering
\begin{tikzpicture}[scale=1,transform shape]
  \Vertex[x=0.5,y=6]{0}
  \Vertex[x=2,y=6]{1}
  \Vertex[x=4,y=6]{2}

  \Vertex[x=-1,y=4]{ini}
  \Vertex[x=3,y=4]{4}
  \Vertex[x=7,y=4]{fim}
  
  \Vertex[x=0.5,y=2]{6}
  \Vertex[x=2,y=2]{7}
  \Vertex[x=4,y=2]{8}
  \Vertex[x=6,y=2]{9}

  \Vertex[x=2,y=0]{10}
  \Vertex[x=4,y=0]{11}

  \tikzstyle{LabelStyle}=[fill=gray!30]
  \tikzstyle{EdgeStyle}=[post]
    \Edge(ini)(0)
    \Edge(0)(1)
    \Edge(1)(2)
    \Edge(2)(fim)
    \Edge(0)(4)
    \Edge(4)(2)
    \Edge(4)(fim)
    \Edge(ini)(6)
    \Edge(6)(7)
    \Edge(7)(4)
    \Edge(7)(8)
    \Edge(8)(fim)
    \Edge(8)(9)
    \Edge(9)(fim)
    \Edge(6)(10)
    \Edge(10)(8)
    \Edge(10)(11)
    \Edge(11)(9)
\end{tikzpicture}
\end{figure}

\chapter{Conclusão}

\end{document}
