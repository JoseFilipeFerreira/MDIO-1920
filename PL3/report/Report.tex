\documentclass[a4paper]{report}
\usepackage[utf8]{inputenc}
\usepackage[portuguese]{babel}
\usepackage{hyperref}
\usepackage{a4wide}
\hypersetup{pdftitle={MDIO - Trabalho 2},
pdfauthor={João Teixeira, José Ferreira, Miguel Solino},
colorlinks=true,
urlcolor=blue,
linkcolor=black}
\usepackage{subcaption}
\usepackage[cache=false]{minted}
\usepackage{listings}
\usepackage{booktabs}
\usepackage{multirow}
\usepackage{appendix}
\usepackage{tikz}
\usepackage{authblk}
\usepackage{bashful}
\usepackage{verbatim}
\usepackage{amsmath}
\usepackage{tikz}
\usepackage{tikz,fullpage}
\usepackage{pgfgantt}
\usetikzlibrary{arrows,%
                petri,%
                topaths}%
\usepackage{tkz-berge}
\usetikzlibrary{positioning,automata,decorations.markings}
\AfterEndEnvironment{figure}{\noindent\ignorespaces}
\AfterEndEnvironment{table}{\noindent\ignorespaces}

\begin{document}

\title{MDIO - Trabalho 2\\ 
\large Grupo Nº 3}
\author{João Teixeira (A85504) \and José Ferreira (A83683) \and Miguel Solino (A86435)}
\date{\today}

\begin{center}
    \begin{minipage}{0.75\linewidth}
        \centering
        \includegraphics[width=0.4\textwidth]{images/eng.jpeg}\par\vspace{1cm}
        \vspace{1.5cm}
        \href{https://www.uminho.pt/PT}
        {\color{black}{\scshape\LARGE Universidade do Minho}} \par
        \vspace{1cm}
        \href{https://www.di.uminho.pt/}
        {\color{black}{\scshape\Large Departamento de Informática}} \par
        \vspace{1.5cm}
        \maketitle
    \end{minipage}
\end{center}

\tableofcontents

\pagebreak

\chapter{Introdução}
O \textit{Critical Path Method} é uma ferramenta fundamental para a gestão de
projetos.

\chapter{Parte 0}
\section{Alínea 1 - Nova rede do projeto}
A lista de atividades a realizar e as suas respectivas precedência apresentadas
inicialmente no enunciado são as seguintes:
\begin{table}[H]
    \centering
    \begin{tabular}{|l|l|l|}
        \hline
        Atividade & Duração & Precedência \\ \hline
        0         & 4       & -           \\ \hline
        1         & 6       & 0           \\ \hline
        2         & 7       & 1, 4        \\ \hline
        3         & 2       & 2, 5        \\ \hline
        4         & 9       & 0, 7        \\ \hline
        5         & 4       & 4, 8        \\ \hline
        6         & 5       & -           \\ \hline
        7         & 6       & 6           \\ \hline
        8         & 4       & 7, 10       \\ \hline
        9         & 2       & 8, 11       \\ \hline
        10        & 8       & 6           \\ \hline
        11        & 7       & 10          \\ \hline
    \end{tabular}
    \caption{projeto inicial}
\end{table}

\begin{figure}[H]
    \centering
    \begin{tikzpicture}[scale=1,transform shape]
        \Vertex[x=0.5,y=6]{0}
        \Vertex[x=2,y=6]{1}
        \Vertex[x=4,y=6]{2}
        \Vertex[x=6,y=6]{3}

        \Vertex[x=-1,y=4]{ini}
        \Vertex[x=3,y=4]{4}
        \Vertex[x=5,y=4]{5}
        \Vertex[x=7,y=4]{fim}
  
        \Vertex[x=0.5,y=2]{6}
        \Vertex[x=2,y=2]{7}
        \Vertex[x=4,y=2]{8}
        \Vertex[x=6,y=2]{9}

        \Vertex[x=2,y=0]{10}
        \Vertex[x=4,y=0]{11}

        \tikzstyle{LabelStyle}=[fill=gray!30]
        \tikzstyle{EdgeStyle}=[post]
            \Edge(ini)(0)
            \Edge(0)(1)
            \Edge(1)(2)
            \Edge(2)(3)
            \Edge(3)(fim)
            \Edge(0)(4)
            \Edge(4)(2)
            \Edge(4)(5)
            \Edge(5)(3)
            \Edge(5)(fim)
            \Edge(ini)(6)
            \Edge(6)(7)
            \Edge(7)(4)
            \Edge(7)(8)
            \Edge(8)(5)
            \Edge(8)(9)
            \Edge(9)(fim)
            \Edge(6)(10)
            \Edge(10)(8)
            \Edge(10)(11)
            \Edge(11)(9)
    \end{tikzpicture}
    \caption{representação do projeto inicial em forma de grafo}
\end{figure}
Observando os números de inscrição dos membros do grupo, constatamos
que o maior pertencia ao aluno Miguel Solino (86435).
Fazendo \textit{pattern matching} para o padrão ABCDE obtivemos os seguintes
valores:

\begin{enumerate}
    \item A é igual a 8;
    \item B é igual a 6;
    \item C é igual a 4;
    \item D é igual a 3;
    \item E é igual a 5;
\end{enumerate}
Assim, seguindo o descrito no enunciado, as atividades com o número 3 e 5 serão
eliminadas.
Seguindo os critérios definidos, a nova tabela representativa das atividades a
realizar é a seguinte:

\begin{table}[H]
    \centering
    \begin{tabular}{|l|l|l|}
        \hline
        Atividade & Duração & Precedência \\ \hline
        0         & 4       & -           \\ \hline
        1         & 6       & 0           \\ \hline
        2         & 7       & 1, 4        \\ \hline
        %3         & 2       & 2, 5        \\ \hline
        4         & 9       & 0, 7        \\ \hline
        %5         & 4       & 4, 8        \\ \hline
        6         & 5       & -           \\ \hline
        7         & 6       & 6           \\ \hline
        8         & 4       & 7, 10       \\ \hline
        9         & 2       & 8, 11       \\ \hline
        10        & 8       & 6           \\ \hline
        11        & 7       & 10          \\ \hline
    \end{tabular}
    \caption{projeto a realizar}
\end{table}

\begin{figure}[H]
    \centering
    \begin{tikzpicture}[scale=1,transform shape]
        \Vertex[x=0.5,y=6]{0}
        \Vertex[x=2,y=6]{1}
        \Vertex[x=4,y=6]{2}

        \Vertex[x=-1,y=4]{ini}
        \Vertex[x=3,y=4]{4}
        \Vertex[x=7,y=4]{fim}
  
        \Vertex[x=0.5,y=2]{6}
        \Vertex[x=2,y=2]{7}
        \Vertex[x=4,y=2]{8}
        \Vertex[x=6,y=2]{9}

        \Vertex[x=2,y=0]{10}
        \Vertex[x=4,y=0]{11}

        \tikzstyle{LabelStyle}=[fill=gray!30]
        \tikzstyle{EdgeStyle}=[post]
            \Edge(ini)(0)
            \Edge(0)(1)
            \Edge(1)(2)
            \Edge(2)(fim)
            \Edge(0)(4)
            \Edge(4)(2)
            \Edge(4)(fim)
            \Edge(ini)(6)
            \Edge(6)(7)
            \Edge(7)(4)
            \Edge(7)(8)
            \Edge(8)(fim)
            \Edge(8)(9)
            \Edge(9)(fim)
            \Edge(6)(10)
            \Edge(10)(8)
            \Edge(10)(11)
            \Edge(11)(9)
    \end{tikzpicture}
    \caption{representação do projeto a realizar em forma de grafo}
\end{figure}

\pagebreak
\section{Alínea 2 - Diagrama de Gantt}
O diagrama de Gantt representa o plano de atividades do projeto.\\
A fim de saber o tempo em que começa cada atividade foi resolvido um problema de
programação linear que respeita as seguintes regras:
\begin{itemize}
    \item As variáveis de decisão são do tipo ti, com i igual ao id de uma
        atividade, que representa do tempo em que começa uma dada atividade.
    \item As restrições garantem as precedências entre as atividades e que as
        suas durações são respeitadas.
    \item A função objetivo consiste em  minimizar a variável de decisão
        \textit{tfinal} que representa o instante em que acaba o projeto.
\end{itemize}

\subsection{Texto de input}
\verbatiminput{gantt.lp}

\pagebreak
\subsection{Ficheiro de output}
\bash[stdout]
lp_solve gantt.lp
\END


\subsection{Interpretação dos resultados}
\begin{figure}[H]
    \centering
    \begin{ganttchart}[vgrid, hgrid]{1}{27}
        \ganttbar{Task 0}{1}{4}\\
        \ganttbar{Task 6}{1}{5}\\

        \ganttbar{Task 1}{5}{10}\\
        \ganttbar{Task 7}{6}{11}\\
        \ganttbar{Task 10}{6}{13}\\
        
        \ganttbar{Task 4}{12}{20}\\
        \ganttbar{Task 11}{14}{20}\\
        
        \ganttbar{Task 8}{17}{20}\\
        
        \ganttbar{Task 9}{21}{22}\\
        
        \ganttbar{Task 2}{21}{27}\\
    \end{ganttchart}
    \caption{diagrama de Gantt}
\end{figure}

\chapter{Parte 1}
\section{Caminho Critico}
O caminho crítico corresponde ao caminho mais longo entre o vértice ini e o
vértice fim, o que corresponde ao início do projecto e  ao fim do projecto.\\
Para obter este caminho é necessário resolver um problema de programação linear
que respeita os seguintes critérios:\\
\begin{itemize}
    \item As variáveis de decisão são do tipo xij, com i a ser o vértice de origem e j o
vértice de destino, e representam o fluxo de um determinado arco.
    \item As restrições garantem que o fluxo que entra num vértice é igual ao fluxo que
sai.
    \item A função objetivo garante que o caminho obtido é de facto o maior
\end{itemize}

\subsection{Texto de input}
\verbatiminput{critical_path.lp}

\pagebreak
\subsection{Ficheiro de output}
\bash[stdout]
lp_solve critical_path.lp
\END

\subsection{Interpretação dos resultados}
Analisando os resultados obtidos concluímos que o caminho mais longo entre o
nodo ini e o nodo fim é o caminho que passa pelos vértices 6, 7, 4 e 2.\\
Assim, o caminho crítico deste projeto corresponde ao caminho 6, 7, 4 e 2.

\section{Atividade paralelas}
No diagrama de Gantt obtido anteriormente é possível observar facilmente
atividades que ocorrem ao mesmo tempo.\\
Neste caso, a atividade 1, 7 e 10 ocorrem as três em simultâneo (sendo que a
atividade 7 pertence ao caminho critico calculado anteriormente).\\
Se existir apenas um equipamento para realizar estas atividades, isso implica
que estas não podem estar a decorrer em simultâneo.

\section{Máquina única}
Se existir uma máquina partilhada entre as atividades 1, 7 e 10 e apenas existir
uma máquina ativa, apenas pode estar a decorrer uma dessas atividades em
simultâneo.

\pagebreak
\subsection{Texto de input}
\verbatiminput{overlap.lp}

\pagebreak
\subsection{Ficheiro de output}
\bash[stdout]
lp_solve overlap.lp
\END

\subsection{Interpretação dos resultados}
\begin{figure}[H]
    \centering
    \begin{ganttchart}[vgrid, hgrid]{1}{32}
        \ganttbar{Task 0}{1}{4}\\
        \ganttbar{Task 6}{1}{5}\\

        \ganttbar{Task 7}{5}{10}\\
        \ganttbar{Task 10}{11}{18}\\
        \ganttbar{Task 1}{19}{24}\\
        
        \ganttbar{Task 4}{16}{24}\\
        \ganttbar{Task 11}{19}{25}\\
        
        \ganttbar{Task 8}{19}{22}\\
        
        \ganttbar{Task 9}{26}{27}\\
        
        \ganttbar{Task 2}{25}{31}\\
    \end{ganttchart}
    \caption{diagrama de Gantt}
\end{figure}


\chapter{Parte 2}
A fim de saber quanto se deve reduzir cada uma das atividades é necessário
seguir as seguintes regras:
\begin{itemize}
    \item As variáveis de decisão são do tipo ti, com i igual ao id de uma
        atividade, que representa do tempo em que começa uma dada atividade.
        rpi representa quanto deve ser reduzido seguindo o primeiro parâmetro.
        rsi representa quanto deve ser reduzido seguindo o segundo parâmetro.
    \item As restrições garantem as precedências entre as atividades e que as
        suas durações são respeitadas.
    \item A função objetivo consiste em  minimizar o custo total das reduções.
\end{itemize}

\pagebreak
\subsection{Texto de input}
\verbatiminput{reducao.lp}

\pagebreak
\subsection{Ficheiro de output}
\bash[stdout]
lp_solve reducao.lp
\END

Infelizmente, não conseguimos gerar as restrições que forçam a que as primeiras
reduções dejam efetuadas em primeiro lugar.
\end{document}
